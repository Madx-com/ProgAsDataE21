\documentclass[11pt]{article}

    \usepackage[breakable]{tcolorbox}
    \usepackage{parskip} % Stop auto-indenting (to mimic markdown behaviour)
    
    \usepackage{iftex}
    \ifPDFTeX
    	\usepackage[T1]{fontenc}
    	\usepackage{mathpazo}
    \else
    	\usepackage{fontspec}
    \fi

    % Basic figure setup, for now with no caption control since it's done
    % automatically by Pandoc (which extracts ![](path) syntax from Markdown).
    \usepackage{graphicx}
    % Maintain compatibility with old templates. Remove in nbconvert 6.0
    \let\Oldincludegraphics\includegraphics
    % Ensure that by default, figures have no caption (until we provide a
    % proper Figure object with a Caption API and a way to capture that
    % in the conversion process - todo).
    \usepackage{caption}
    \DeclareCaptionFormat{nocaption}{}
    \captionsetup{format=nocaption,aboveskip=0pt,belowskip=0pt}

    \usepackage{float}
    \floatplacement{figure}{H} % forces figures to be placed at the correct location
    \usepackage{xcolor} % Allow colors to be defined
    \usepackage{enumerate} % Needed for markdown enumerations to work
    \usepackage{geometry} % Used to adjust the document margins
    \usepackage{amsmath} % Equations
    \usepackage{amssymb} % Equations
    \usepackage{textcomp} % defines textquotesingle
    % Hack from http://tex.stackexchange.com/a/47451/13684:
    \AtBeginDocument{%
        \def\PYZsq{\textquotesingle}% Upright quotes in Pygmentized code
    }
    \usepackage{upquote} % Upright quotes for verbatim code
    \usepackage{eurosym} % defines \euro
    \usepackage[mathletters]{ucs} % Extended unicode (utf-8) support
    \usepackage{fancyvrb} % verbatim replacement that allows latex
    \usepackage{grffile} % extends the file name processing of package graphics 
                         % to support a larger range
    \makeatletter % fix for old versions of grffile with XeLaTeX
    \@ifpackagelater{grffile}{2019/11/01}
    {
      % Do nothing on new versions
    }
    {
      \def\Gread@@xetex#1{%
        \IfFileExists{"\Gin@base".bb}%
        {\Gread@eps{\Gin@base.bb}}%
        {\Gread@@xetex@aux#1}%
      }
    }
    \makeatother
    \usepackage[Export]{adjustbox} % Used to constrain images to a maximum size
    \adjustboxset{max size={0.9\linewidth}{0.9\paperheight}}

    % The hyperref package gives us a pdf with properly built
    % internal navigation ('pdf bookmarks' for the table of contents,
    % internal cross-reference links, web links for URLs, etc.)
    \usepackage{hyperref}
    % The default LaTeX title has an obnoxious amount of whitespace. By default,
    % titling removes some of it. It also provides customization options.
    \usepackage{titling}
    \usepackage{longtable} % longtable support required by pandoc >1.10
    \usepackage{booktabs}  % table support for pandoc > 1.12.2
    \usepackage[inline]{enumitem} % IRkernel/repr support (it uses the enumerate* environment)
    \usepackage[normalem]{ulem} % ulem is needed to support strikethroughs (\sout)
                                % normalem makes italics be italics, not underlines
    \usepackage{mathrsfs}
    

    
    % Colors for the hyperref package
    \definecolor{urlcolor}{rgb}{0,.145,.698}
    \definecolor{linkcolor}{rgb}{.71,0.21,0.01}
    \definecolor{citecolor}{rgb}{.12,.54,.11}

    % ANSI colors
    \definecolor{ansi-black}{HTML}{3E424D}
    \definecolor{ansi-black-intense}{HTML}{282C36}
    \definecolor{ansi-red}{HTML}{E75C58}
    \definecolor{ansi-red-intense}{HTML}{B22B31}
    \definecolor{ansi-green}{HTML}{00A250}
    \definecolor{ansi-green-intense}{HTML}{007427}
    \definecolor{ansi-yellow}{HTML}{DDB62B}
    \definecolor{ansi-yellow-intense}{HTML}{B27D12}
    \definecolor{ansi-blue}{HTML}{208FFB}
    \definecolor{ansi-blue-intense}{HTML}{0065CA}
    \definecolor{ansi-magenta}{HTML}{D160C4}
    \definecolor{ansi-magenta-intense}{HTML}{A03196}
    \definecolor{ansi-cyan}{HTML}{60C6C8}
    \definecolor{ansi-cyan-intense}{HTML}{258F8F}
    \definecolor{ansi-white}{HTML}{C5C1B4}
    \definecolor{ansi-white-intense}{HTML}{A1A6B2}
    \definecolor{ansi-default-inverse-fg}{HTML}{FFFFFF}
    \definecolor{ansi-default-inverse-bg}{HTML}{000000}

    % common color for the border for error outputs.
    \definecolor{outerrorbackground}{HTML}{FFDFDF}

    % commands and environments needed by pandoc snippets
    % extracted from the output of `pandoc -s`
    \providecommand{\tightlist}{%
      \setlength{\itemsep}{0pt}\setlength{\parskip}{0pt}}
    \DefineVerbatimEnvironment{Highlighting}{Verbatim}{commandchars=\\\{\}}
    % Add ',fontsize=\small' for more characters per line
    \newenvironment{Shaded}{}{}
    \newcommand{\KeywordTok}[1]{\textcolor[rgb]{0.00,0.44,0.13}{\textbf{{#1}}}}
    \newcommand{\DataTypeTok}[1]{\textcolor[rgb]{0.56,0.13,0.00}{{#1}}}
    \newcommand{\DecValTok}[1]{\textcolor[rgb]{0.25,0.63,0.44}{{#1}}}
    \newcommand{\BaseNTok}[1]{\textcolor[rgb]{0.25,0.63,0.44}{{#1}}}
    \newcommand{\FloatTok}[1]{\textcolor[rgb]{0.25,0.63,0.44}{{#1}}}
    \newcommand{\CharTok}[1]{\textcolor[rgb]{0.25,0.44,0.63}{{#1}}}
    \newcommand{\StringTok}[1]{\textcolor[rgb]{0.25,0.44,0.63}{{#1}}}
    \newcommand{\CommentTok}[1]{\textcolor[rgb]{0.38,0.63,0.69}{\textit{{#1}}}}
    \newcommand{\OtherTok}[1]{\textcolor[rgb]{0.00,0.44,0.13}{{#1}}}
    \newcommand{\AlertTok}[1]{\textcolor[rgb]{1.00,0.00,0.00}{\textbf{{#1}}}}
    \newcommand{\FunctionTok}[1]{\textcolor[rgb]{0.02,0.16,0.49}{{#1}}}
    \newcommand{\RegionMarkerTok}[1]{{#1}}
    \newcommand{\ErrorTok}[1]{\textcolor[rgb]{1.00,0.00,0.00}{\textbf{{#1}}}}
    \newcommand{\NormalTok}[1]{{#1}}
    
    % Additional commands for more recent versions of Pandoc
    \newcommand{\ConstantTok}[1]{\textcolor[rgb]{0.53,0.00,0.00}{{#1}}}
    \newcommand{\SpecialCharTok}[1]{\textcolor[rgb]{0.25,0.44,0.63}{{#1}}}
    \newcommand{\VerbatimStringTok}[1]{\textcolor[rgb]{0.25,0.44,0.63}{{#1}}}
    \newcommand{\SpecialStringTok}[1]{\textcolor[rgb]{0.73,0.40,0.53}{{#1}}}
    \newcommand{\ImportTok}[1]{{#1}}
    \newcommand{\DocumentationTok}[1]{\textcolor[rgb]{0.73,0.13,0.13}{\textit{{#1}}}}
    \newcommand{\AnnotationTok}[1]{\textcolor[rgb]{0.38,0.63,0.69}{\textbf{\textit{{#1}}}}}
    \newcommand{\CommentVarTok}[1]{\textcolor[rgb]{0.38,0.63,0.69}{\textbf{\textit{{#1}}}}}
    \newcommand{\VariableTok}[1]{\textcolor[rgb]{0.10,0.09,0.49}{{#1}}}
    \newcommand{\ControlFlowTok}[1]{\textcolor[rgb]{0.00,0.44,0.13}{\textbf{{#1}}}}
    \newcommand{\OperatorTok}[1]{\textcolor[rgb]{0.40,0.40,0.40}{{#1}}}
    \newcommand{\BuiltInTok}[1]{{#1}}
    \newcommand{\ExtensionTok}[1]{{#1}}
    \newcommand{\PreprocessorTok}[1]{\textcolor[rgb]{0.74,0.48,0.00}{{#1}}}
    \newcommand{\AttributeTok}[1]{\textcolor[rgb]{0.49,0.56,0.16}{{#1}}}
    \newcommand{\InformationTok}[1]{\textcolor[rgb]{0.38,0.63,0.69}{\textbf{\textit{{#1}}}}}
    \newcommand{\WarningTok}[1]{\textcolor[rgb]{0.38,0.63,0.69}{\textbf{\textit{{#1}}}}}
    
    
    % Define a nice break command that doesn't care if a line doesn't already
    % exist.
    \def\br{\hspace*{\fill} \\* }
    % Math Jax compatibility definitions
    \def\gt{>}
    \def\lt{<}
    \let\Oldtex\TeX
    \let\Oldlatex\LaTeX
    \renewcommand{\TeX}{\textrm{\Oldtex}}
    \renewcommand{\LaTeX}{\textrm{\Oldlatex}}
    % Document parameters
    % Document title
    \title{assignment1py}
    
    
    
    
    
% Pygments definitions
\makeatletter
\def\PY@reset{\let\PY@it=\relax \let\PY@bf=\relax%
    \let\PY@ul=\relax \let\PY@tc=\relax%
    \let\PY@bc=\relax \let\PY@ff=\relax}
\def\PY@tok#1{\csname PY@tok@#1\endcsname}
\def\PY@toks#1+{\ifx\relax#1\empty\else%
    \PY@tok{#1}\expandafter\PY@toks\fi}
\def\PY@do#1{\PY@bc{\PY@tc{\PY@ul{%
    \PY@it{\PY@bf{\PY@ff{#1}}}}}}}
\def\PY#1#2{\PY@reset\PY@toks#1+\relax+\PY@do{#2}}

\@namedef{PY@tok@w}{\def\PY@tc##1{\textcolor[rgb]{0.73,0.73,0.73}{##1}}}
\@namedef{PY@tok@c}{\let\PY@it=\textit\def\PY@tc##1{\textcolor[rgb]{0.25,0.50,0.50}{##1}}}
\@namedef{PY@tok@cp}{\def\PY@tc##1{\textcolor[rgb]{0.74,0.48,0.00}{##1}}}
\@namedef{PY@tok@k}{\let\PY@bf=\textbf\def\PY@tc##1{\textcolor[rgb]{0.00,0.50,0.00}{##1}}}
\@namedef{PY@tok@kp}{\def\PY@tc##1{\textcolor[rgb]{0.00,0.50,0.00}{##1}}}
\@namedef{PY@tok@kt}{\def\PY@tc##1{\textcolor[rgb]{0.69,0.00,0.25}{##1}}}
\@namedef{PY@tok@o}{\def\PY@tc##1{\textcolor[rgb]{0.40,0.40,0.40}{##1}}}
\@namedef{PY@tok@ow}{\let\PY@bf=\textbf\def\PY@tc##1{\textcolor[rgb]{0.67,0.13,1.00}{##1}}}
\@namedef{PY@tok@nb}{\def\PY@tc##1{\textcolor[rgb]{0.00,0.50,0.00}{##1}}}
\@namedef{PY@tok@nf}{\def\PY@tc##1{\textcolor[rgb]{0.00,0.00,1.00}{##1}}}
\@namedef{PY@tok@nc}{\let\PY@bf=\textbf\def\PY@tc##1{\textcolor[rgb]{0.00,0.00,1.00}{##1}}}
\@namedef{PY@tok@nn}{\let\PY@bf=\textbf\def\PY@tc##1{\textcolor[rgb]{0.00,0.00,1.00}{##1}}}
\@namedef{PY@tok@ne}{\let\PY@bf=\textbf\def\PY@tc##1{\textcolor[rgb]{0.82,0.25,0.23}{##1}}}
\@namedef{PY@tok@nv}{\def\PY@tc##1{\textcolor[rgb]{0.10,0.09,0.49}{##1}}}
\@namedef{PY@tok@no}{\def\PY@tc##1{\textcolor[rgb]{0.53,0.00,0.00}{##1}}}
\@namedef{PY@tok@nl}{\def\PY@tc##1{\textcolor[rgb]{0.63,0.63,0.00}{##1}}}
\@namedef{PY@tok@ni}{\let\PY@bf=\textbf\def\PY@tc##1{\textcolor[rgb]{0.60,0.60,0.60}{##1}}}
\@namedef{PY@tok@na}{\def\PY@tc##1{\textcolor[rgb]{0.49,0.56,0.16}{##1}}}
\@namedef{PY@tok@nt}{\let\PY@bf=\textbf\def\PY@tc##1{\textcolor[rgb]{0.00,0.50,0.00}{##1}}}
\@namedef{PY@tok@nd}{\def\PY@tc##1{\textcolor[rgb]{0.67,0.13,1.00}{##1}}}
\@namedef{PY@tok@s}{\def\PY@tc##1{\textcolor[rgb]{0.73,0.13,0.13}{##1}}}
\@namedef{PY@tok@sd}{\let\PY@it=\textit\def\PY@tc##1{\textcolor[rgb]{0.73,0.13,0.13}{##1}}}
\@namedef{PY@tok@si}{\let\PY@bf=\textbf\def\PY@tc##1{\textcolor[rgb]{0.73,0.40,0.53}{##1}}}
\@namedef{PY@tok@se}{\let\PY@bf=\textbf\def\PY@tc##1{\textcolor[rgb]{0.73,0.40,0.13}{##1}}}
\@namedef{PY@tok@sr}{\def\PY@tc##1{\textcolor[rgb]{0.73,0.40,0.53}{##1}}}
\@namedef{PY@tok@ss}{\def\PY@tc##1{\textcolor[rgb]{0.10,0.09,0.49}{##1}}}
\@namedef{PY@tok@sx}{\def\PY@tc##1{\textcolor[rgb]{0.00,0.50,0.00}{##1}}}
\@namedef{PY@tok@m}{\def\PY@tc##1{\textcolor[rgb]{0.40,0.40,0.40}{##1}}}
\@namedef{PY@tok@gh}{\let\PY@bf=\textbf\def\PY@tc##1{\textcolor[rgb]{0.00,0.00,0.50}{##1}}}
\@namedef{PY@tok@gu}{\let\PY@bf=\textbf\def\PY@tc##1{\textcolor[rgb]{0.50,0.00,0.50}{##1}}}
\@namedef{PY@tok@gd}{\def\PY@tc##1{\textcolor[rgb]{0.63,0.00,0.00}{##1}}}
\@namedef{PY@tok@gi}{\def\PY@tc##1{\textcolor[rgb]{0.00,0.63,0.00}{##1}}}
\@namedef{PY@tok@gr}{\def\PY@tc##1{\textcolor[rgb]{1.00,0.00,0.00}{##1}}}
\@namedef{PY@tok@ge}{\let\PY@it=\textit}
\@namedef{PY@tok@gs}{\let\PY@bf=\textbf}
\@namedef{PY@tok@gp}{\let\PY@bf=\textbf\def\PY@tc##1{\textcolor[rgb]{0.00,0.00,0.50}{##1}}}
\@namedef{PY@tok@go}{\def\PY@tc##1{\textcolor[rgb]{0.53,0.53,0.53}{##1}}}
\@namedef{PY@tok@gt}{\def\PY@tc##1{\textcolor[rgb]{0.00,0.27,0.87}{##1}}}
\@namedef{PY@tok@err}{\def\PY@bc##1{{\setlength{\fboxsep}{\string -\fboxrule}\fcolorbox[rgb]{1.00,0.00,0.00}{1,1,1}{\strut ##1}}}}
\@namedef{PY@tok@kc}{\let\PY@bf=\textbf\def\PY@tc##1{\textcolor[rgb]{0.00,0.50,0.00}{##1}}}
\@namedef{PY@tok@kd}{\let\PY@bf=\textbf\def\PY@tc##1{\textcolor[rgb]{0.00,0.50,0.00}{##1}}}
\@namedef{PY@tok@kn}{\let\PY@bf=\textbf\def\PY@tc##1{\textcolor[rgb]{0.00,0.50,0.00}{##1}}}
\@namedef{PY@tok@kr}{\let\PY@bf=\textbf\def\PY@tc##1{\textcolor[rgb]{0.00,0.50,0.00}{##1}}}
\@namedef{PY@tok@bp}{\def\PY@tc##1{\textcolor[rgb]{0.00,0.50,0.00}{##1}}}
\@namedef{PY@tok@fm}{\def\PY@tc##1{\textcolor[rgb]{0.00,0.00,1.00}{##1}}}
\@namedef{PY@tok@vc}{\def\PY@tc##1{\textcolor[rgb]{0.10,0.09,0.49}{##1}}}
\@namedef{PY@tok@vg}{\def\PY@tc##1{\textcolor[rgb]{0.10,0.09,0.49}{##1}}}
\@namedef{PY@tok@vi}{\def\PY@tc##1{\textcolor[rgb]{0.10,0.09,0.49}{##1}}}
\@namedef{PY@tok@vm}{\def\PY@tc##1{\textcolor[rgb]{0.10,0.09,0.49}{##1}}}
\@namedef{PY@tok@sa}{\def\PY@tc##1{\textcolor[rgb]{0.73,0.13,0.13}{##1}}}
\@namedef{PY@tok@sb}{\def\PY@tc##1{\textcolor[rgb]{0.73,0.13,0.13}{##1}}}
\@namedef{PY@tok@sc}{\def\PY@tc##1{\textcolor[rgb]{0.73,0.13,0.13}{##1}}}
\@namedef{PY@tok@dl}{\def\PY@tc##1{\textcolor[rgb]{0.73,0.13,0.13}{##1}}}
\@namedef{PY@tok@s2}{\def\PY@tc##1{\textcolor[rgb]{0.73,0.13,0.13}{##1}}}
\@namedef{PY@tok@sh}{\def\PY@tc##1{\textcolor[rgb]{0.73,0.13,0.13}{##1}}}
\@namedef{PY@tok@s1}{\def\PY@tc##1{\textcolor[rgb]{0.73,0.13,0.13}{##1}}}
\@namedef{PY@tok@mb}{\def\PY@tc##1{\textcolor[rgb]{0.40,0.40,0.40}{##1}}}
\@namedef{PY@tok@mf}{\def\PY@tc##1{\textcolor[rgb]{0.40,0.40,0.40}{##1}}}
\@namedef{PY@tok@mh}{\def\PY@tc##1{\textcolor[rgb]{0.40,0.40,0.40}{##1}}}
\@namedef{PY@tok@mi}{\def\PY@tc##1{\textcolor[rgb]{0.40,0.40,0.40}{##1}}}
\@namedef{PY@tok@il}{\def\PY@tc##1{\textcolor[rgb]{0.40,0.40,0.40}{##1}}}
\@namedef{PY@tok@mo}{\def\PY@tc##1{\textcolor[rgb]{0.40,0.40,0.40}{##1}}}
\@namedef{PY@tok@ch}{\let\PY@it=\textit\def\PY@tc##1{\textcolor[rgb]{0.25,0.50,0.50}{##1}}}
\@namedef{PY@tok@cm}{\let\PY@it=\textit\def\PY@tc##1{\textcolor[rgb]{0.25,0.50,0.50}{##1}}}
\@namedef{PY@tok@cpf}{\let\PY@it=\textit\def\PY@tc##1{\textcolor[rgb]{0.25,0.50,0.50}{##1}}}
\@namedef{PY@tok@c1}{\let\PY@it=\textit\def\PY@tc##1{\textcolor[rgb]{0.25,0.50,0.50}{##1}}}
\@namedef{PY@tok@cs}{\let\PY@it=\textit\def\PY@tc##1{\textcolor[rgb]{0.25,0.50,0.50}{##1}}}

\def\PYZbs{\char`\\}
\def\PYZus{\char`\_}
\def\PYZob{\char`\{}
\def\PYZcb{\char`\}}
\def\PYZca{\char`\^}
\def\PYZam{\char`\&}
\def\PYZlt{\char`\<}
\def\PYZgt{\char`\>}
\def\PYZsh{\char`\#}
\def\PYZpc{\char`\%}
\def\PYZdl{\char`\$}
\def\PYZhy{\char`\-}
\def\PYZsq{\char`\'}
\def\PYZdq{\char`\"}
\def\PYZti{\char`\~}
% for compatibility with earlier versions
\def\PYZat{@}
\def\PYZlb{[}
\def\PYZrb{]}
\makeatother


    % For linebreaks inside Verbatim environment from package fancyvrb. 
    \makeatletter
        \newbox\Wrappedcontinuationbox 
        \newbox\Wrappedvisiblespacebox 
        \newcommand*\Wrappedvisiblespace {\textcolor{red}{\textvisiblespace}} 
        \newcommand*\Wrappedcontinuationsymbol {\textcolor{red}{\llap{\tiny$\m@th\hookrightarrow$}}} 
        \newcommand*\Wrappedcontinuationindent {3ex } 
        \newcommand*\Wrappedafterbreak {\kern\Wrappedcontinuationindent\copy\Wrappedcontinuationbox} 
        % Take advantage of the already applied Pygments mark-up to insert 
        % potential linebreaks for TeX processing. 
        %        {, <, #, %, $, ' and ": go to next line. 
        %        _, }, ^, &, >, - and ~: stay at end of broken line. 
        % Use of \textquotesingle for straight quote. 
        \newcommand*\Wrappedbreaksatspecials {% 
            \def\PYGZus{\discretionary{\char`\_}{\Wrappedafterbreak}{\char`\_}}% 
            \def\PYGZob{\discretionary{}{\Wrappedafterbreak\char`\{}{\char`\{}}% 
            \def\PYGZcb{\discretionary{\char`\}}{\Wrappedafterbreak}{\char`\}}}% 
            \def\PYGZca{\discretionary{\char`\^}{\Wrappedafterbreak}{\char`\^}}% 
            \def\PYGZam{\discretionary{\char`\&}{\Wrappedafterbreak}{\char`\&}}% 
            \def\PYGZlt{\discretionary{}{\Wrappedafterbreak\char`\<}{\char`\<}}% 
            \def\PYGZgt{\discretionary{\char`\>}{\Wrappedafterbreak}{\char`\>}}% 
            \def\PYGZsh{\discretionary{}{\Wrappedafterbreak\char`\#}{\char`\#}}% 
            \def\PYGZpc{\discretionary{}{\Wrappedafterbreak\char`\%}{\char`\%}}% 
            \def\PYGZdl{\discretionary{}{\Wrappedafterbreak\char`\$}{\char`\$}}% 
            \def\PYGZhy{\discretionary{\char`\-}{\Wrappedafterbreak}{\char`\-}}% 
            \def\PYGZsq{\discretionary{}{\Wrappedafterbreak\textquotesingle}{\textquotesingle}}% 
            \def\PYGZdq{\discretionary{}{\Wrappedafterbreak\char`\"}{\char`\"}}% 
            \def\PYGZti{\discretionary{\char`\~}{\Wrappedafterbreak}{\char`\~}}% 
        } 
        % Some characters . , ; ? ! / are not pygmentized. 
        % This macro makes them "active" and they will insert potential linebreaks 
        \newcommand*\Wrappedbreaksatpunct {% 
            \lccode`\~`\.\lowercase{\def~}{\discretionary{\hbox{\char`\.}}{\Wrappedafterbreak}{\hbox{\char`\.}}}% 
            \lccode`\~`\,\lowercase{\def~}{\discretionary{\hbox{\char`\,}}{\Wrappedafterbreak}{\hbox{\char`\,}}}% 
            \lccode`\~`\;\lowercase{\def~}{\discretionary{\hbox{\char`\;}}{\Wrappedafterbreak}{\hbox{\char`\;}}}% 
            \lccode`\~`\:\lowercase{\def~}{\discretionary{\hbox{\char`\:}}{\Wrappedafterbreak}{\hbox{\char`\:}}}% 
            \lccode`\~`\?\lowercase{\def~}{\discretionary{\hbox{\char`\?}}{\Wrappedafterbreak}{\hbox{\char`\?}}}% 
            \lccode`\~`\!\lowercase{\def~}{\discretionary{\hbox{\char`\!}}{\Wrappedafterbreak}{\hbox{\char`\!}}}% 
            \lccode`\~`\/\lowercase{\def~}{\discretionary{\hbox{\char`\/}}{\Wrappedafterbreak}{\hbox{\char`\/}}}% 
            \catcode`\.\active
            \catcode`\,\active 
            \catcode`\;\active
            \catcode`\:\active
            \catcode`\?\active
            \catcode`\!\active
            \catcode`\/\active 
            \lccode`\~`\~ 	
        }
    \makeatother

    \let\OriginalVerbatim=\Verbatim
    \makeatletter
    \renewcommand{\Verbatim}[1][1]{%
        %\parskip\z@skip
        \sbox\Wrappedcontinuationbox {\Wrappedcontinuationsymbol}%
        \sbox\Wrappedvisiblespacebox {\FV@SetupFont\Wrappedvisiblespace}%
        \def\FancyVerbFormatLine ##1{\hsize\linewidth
            \vtop{\raggedright\hyphenpenalty\z@\exhyphenpenalty\z@
                \doublehyphendemerits\z@\finalhyphendemerits\z@
                \strut ##1\strut}%
        }%
        % If the linebreak is at a space, the latter will be displayed as visible
        % space at end of first line, and a continuation symbol starts next line.
        % Stretch/shrink are however usually zero for typewriter font.
        \def\FV@Space {%
            \nobreak\hskip\z@ plus\fontdimen3\font minus\fontdimen4\font
            \discretionary{\copy\Wrappedvisiblespacebox}{\Wrappedafterbreak}
            {\kern\fontdimen2\font}%
        }%
        
        % Allow breaks at special characters using \PYG... macros.
        \Wrappedbreaksatspecials
        % Breaks at punctuation characters . , ; ? ! and / need catcode=\active 	
        \OriginalVerbatim[#1,codes*=\Wrappedbreaksatpunct]%
    }
    \makeatother

    % Exact colors from NB
    \definecolor{incolor}{HTML}{303F9F}
    \definecolor{outcolor}{HTML}{D84315}
    \definecolor{cellborder}{HTML}{CFCFCF}
    \definecolor{cellbackground}{HTML}{F7F7F7}
    
    % prompt
    \makeatletter
    \newcommand{\boxspacing}{\kern\kvtcb@left@rule\kern\kvtcb@boxsep}
    \makeatother
    \newcommand{\prompt}[4]{
        {\ttfamily\llap{{\color{#2}[#3]:\hspace{3pt}#4}}\vspace{-\baselineskip}}
    }
    

    
    % Prevent overflowing lines due to hard-to-break entities
    \sloppy 
    % Setup hyperref package
    \hypersetup{
      breaklinks=true,  % so long urls are correctly broken across lines
      colorlinks=true,
      urlcolor=urlcolor,
      linkcolor=linkcolor,
      citecolor=citecolor,
      }
    % Slightly bigger margins than the latex defaults
    
    \geometry{verbose,tmargin=1in,bmargin=1in,lmargin=1in,rmargin=1in}
    
    

\begin{document}
    
    \maketitle
    
    

    
    Programmer som Data - Assignment 1

PLC: 1.1, 1.2, 1.4, 2.1, 2.2, 2.3 (optionally also 2.6).

    \begin{tcolorbox}[breakable, size=fbox, boxrule=1pt, pad at break*=1mm,colback=cellbackground, colframe=cellborder]
\prompt{In}{incolor}{ }{\boxspacing}
\begin{Verbatim}[commandchars=\\\{\}]
\PY{c+c1}{// needed to run F\PYZsh{} Lexer/Parser}
\PY{err}{\PYZsh{}}\PY{n}{r} \PY{l+s}{\PYZdq{}nuget:FsLexYacc,10.0.0\PYZdq{}}

\PY{n}{open} \PY{n}{System}
\PY{err}{\PYZsh{}}\PY{n}{load} \PY{l+s}{\PYZdq{}Intro/intro2.fs\PYZdq{}}
\PY{n}{open} \PY{n}{Intro2}

\PY{n}{printfn} \PY{l+s}{\PYZdq{}\PYZpc{}i\PYZdq{}} \PY{n}{e3v}

\PY{k}{let} \PY{n}{x} \PY{p}{=} \PY{l+m}{4}\PY{l+m}{2}
\end{Verbatim}
\end{tcolorbox}

    
    
    \begin{Verbatim}[commandchars=\\\{\}]
1002
    \end{Verbatim}

    Exercise 1.1 (i) Extend the eval function to handle three additional
operators: max, min, ==. using the Prim absstract syntax taking 2
argument expressions. equals should return 1 when true and 0 when false.

    \begin{tcolorbox}[breakable, size=fbox, boxrule=1pt, pad at break*=1mm,colback=cellbackground, colframe=cellborder]
\prompt{In}{incolor}{ }{\boxspacing}
\begin{Verbatim}[commandchars=\\\{\}]
\PY{k}{let} \PY{n}{rec} \PY{n}{eval} \PY{n+nf}{e} \PY{p}{(}\PY{n}{env} \PY{p}{:} \PY{p}{(}\PY{k+kt}{string} \PY{p}{*} \PY{k+kt}{int}\PY{p}{)} \PY{n}{list}\PY{p}{)} \PY{p}{:} \PY{k+kt}{int} \PY{p}{=}
    \PY{n}{match} \PY{n}{e} \PY{n}{with}
    \PY{p}{|} \PY{n}{CstI} \PY{n}{i}              \PY{p}{\PYZhy{}}\PY{p}{\PYZgt{}} \PY{n}{i}
    \PY{p}{|} \PY{n}{Var} \PY{n}{x}               \PY{p}{\PYZhy{}}\PY{p}{\PYZgt{}} \PY{n}{lookup} \PY{n}{env} \PY{n}{x} 
    \PY{p}{|} \PY{n}{Prim}\PY{p}{(}\PY{l+s}{\PYZdq{}+\PYZdq{}}\PY{p}{,} \PY{n}{e1}\PY{p}{,} \PY{n}{e2}\PY{p}{)}   \PY{p}{\PYZhy{}}\PY{p}{\PYZgt{}} \PY{n}{eval} \PY{n}{e1} \PY{n}{env} \PY{p}{+} \PY{n}{eval} \PY{n}{e2} \PY{n}{env}
    \PY{p}{|} \PY{n}{Prim}\PY{p}{(}\PY{l+s}{\PYZdq{}*\PYZdq{}}\PY{p}{,} \PY{n}{e1}\PY{p}{,} \PY{n}{e2}\PY{p}{)}   \PY{p}{\PYZhy{}}\PY{p}{\PYZgt{}} \PY{n}{eval} \PY{n}{e1} \PY{n}{env} \PY{p}{*} \PY{n}{eval} \PY{n}{e2} \PY{n}{env}
    \PY{p}{|} \PY{n}{Prim}\PY{p}{(}\PY{l+s}{\PYZdq{}\PYZhy{}\PYZdq{}}\PY{p}{,} \PY{n}{e1}\PY{p}{,} \PY{n}{e2}\PY{p}{)}   \PY{p}{\PYZhy{}}\PY{p}{\PYZgt{}} \PY{n}{eval} \PY{n}{e1} \PY{n}{env} \PY{p}{\PYZhy{}} \PY{n}{eval} \PY{n}{e2} \PY{n}{env}
    \PY{p}{|} \PY{n}{Prim}\PY{p}{(}\PY{l+s}{\PYZdq{}max\PYZdq{}}\PY{p}{,} \PY{n}{e1}\PY{p}{,} \PY{n}{e2}\PY{p}{)} \PY{p}{\PYZhy{}}\PY{p}{\PYZgt{}} 
        \PY{k}{let} \PY{n}{r1} \PY{p}{=} \PY{n}{eval} \PY{n}{e1} \PY{n}{env}
        \PY{k}{let} \PY{n}{r2} \PY{p}{=} \PY{n}{eval} \PY{n}{e2} \PY{n}{env}
        \PY{k}{if} \PY{n}{r1} \PY{p}{\PYZgt{}} \PY{n}{r2} \PY{n}{then} \PY{n}{r1} \PY{k}{else} \PY{n}{r2}
    \PY{p}{|} \PY{n}{Prim}\PY{p}{(}\PY{l+s}{\PYZdq{}min\PYZdq{}}\PY{p}{,} \PY{n}{e1}\PY{p}{,} \PY{n}{e2}\PY{p}{)} \PY{p}{\PYZhy{}}\PY{p}{\PYZgt{}} 
        \PY{k}{let} \PY{n}{r1} \PY{p}{=} \PY{n}{eval} \PY{n}{e1} \PY{n}{env}
        \PY{k}{let} \PY{n}{r2} \PY{p}{=} \PY{n}{eval} \PY{n}{e2} \PY{n}{env}
        \PY{k}{if} \PY{n}{r1} \PY{p}{\PYZlt{}} \PY{n}{r2} \PY{n}{then} \PY{n}{r1} \PY{k}{else} \PY{n}{r2}
    \PY{p}{|} \PY{n}{Prim}\PY{p}{(}\PY{l+s}{\PYZdq{}==\PYZdq{}}\PY{p}{,} \PY{n}{e1}\PY{p}{,} \PY{n}{e2}\PY{p}{)} \PY{p}{\PYZhy{}}\PY{p}{\PYZgt{}} 
        \PY{k}{let} \PY{n}{r1} \PY{p}{=} \PY{n}{eval} \PY{n}{e1} \PY{n}{env}
        \PY{k}{let} \PY{n}{r2} \PY{p}{=} \PY{n}{eval} \PY{n}{e2} \PY{n}{env}
        \PY{k}{if} \PY{n}{r1} \PY{p}{=} \PY{n}{r2} \PY{n}{then} \PY{l+m}{1} \PY{k}{else} \PY{l+m}{0}
    \PY{p}{|} \PY{n}{Prim} \PY{n}{\PYZus{}}            \PY{p}{\PYZhy{}}\PY{p}{\PYZgt{}} \PY{n}{failwith} \PY{l+s}{\PYZdq{}unknown primitive\PYZdq{}}
\end{Verbatim}
\end{tcolorbox}

    \begin{enumerate}
\def\labelenumi{(\roman{enumi})}
\setcounter{enumi}{1}
\tightlist
\item
  Write some example expressions in this extended expression language,
  using abstract syntax, and evaluate them using your new eval function.
\end{enumerate}

    \begin{tcolorbox}[breakable, size=fbox, boxrule=1pt, pad at break*=1mm,colback=cellbackground, colframe=cellborder]
\prompt{In}{incolor}{ }{\boxspacing}
\begin{Verbatim}[commandchars=\\\{\}]
\PY{c+c1}{// setting up easier printing..}
\PY{k}{let} \PY{n}{p} \PY{n}{q} \PY{p}{=} \PY{n}{printf} \PY{l+s}{\PYZdq{}\PYZpc{}O\PYZdq{}} \PY{n}{q}
\PY{k}{let} \PY{n+nf}{e} \PY{p}{(}\PY{n}{op}\PY{p}{:}\PY{k+kt}{string}\PY{p}{)} \PY{n}{w} \PY{n}{en} \PY{p}{=} \PY{n}{display}\PY{p}{(} \PY{n}{sprintf} \PY{l+s}{\PYZdq{}Evaluated using \PYZpc{}s: \PYZpc{}A\PYZdq{}} \PY{n}{op} \PY{p}{(}\PY{n}{eval} \PY{n}{w} \PY{n}{en}\PY{p}{)}\PY{p}{)}


\PY{k}{let} \PY{n}{e4} \PY{p}{=} \PY{n}{Prim}\PY{p}{(}\PY{l+s}{\PYZdq{}min\PYZdq{}}\PY{p}{,} \PY{n}{CstI} \PY{l+m}{1}\PY{p}{,} \PY{n}{CstI} \PY{l+m}{1}\PY{l+m}{0}\PY{p}{)}

\PY{n}{p} \PY{n}{e4}
\PY{n}{e} \PY{l+s}{\PYZdq{}min\PYZdq{}} \PY{n}{e4} \PY{p}{[}\PY{p}{]}

\PY{k}{let} \PY{n}{e5} \PY{p}{=} \PY{n}{Prim}\PY{p}{(}\PY{l+s}{\PYZdq{}max\PYZdq{}}\PY{p}{,} \PY{n}{CstI} \PY{l+m}{1}\PY{p}{,} \PY{n}{CstI} \PY{l+m}{1}\PY{l+m}{0}\PY{p}{)}

\PY{n}{p} \PY{n}{e5}
\PY{n}{e} \PY{l+s}{\PYZdq{}max\PYZdq{}} \PY{n}{e5} \PY{p}{[}\PY{p}{]}

\PY{k}{let} \PY{n}{e6} \PY{p}{=} \PY{n}{Prim}\PY{p}{(}\PY{l+s}{\PYZdq{}==\PYZdq{}}\PY{p}{,} \PY{n}{CstI} \PY{l+m}{6}\PY{l+m}{6}\PY{p}{,} \PY{n}{CstI} \PY{l+m}{6}\PY{l+m}{6}\PY{p}{)}

\PY{n}{p} \PY{n}{e6}
\PY{n}{e} \PY{l+s}{\PYZdq{}==\PYZdq{}} \PY{n}{e6} \PY{p}{[}\PY{p}{]}

\PY{k}{let} \PY{n}{e7} \PY{p}{=} \PY{n}{Prim}\PY{p}{(}\PY{l+s}{\PYZdq{}==\PYZdq{}}\PY{p}{,} \PY{n}{CstI} \PY{l+m}{6}\PY{l+m}{6}\PY{p}{,} \PY{n}{CstI} \PY{l+m}{2}\PY{l+m}{2}\PY{p}{)}

\PY{n}{p} \PY{n}{e7}
\PY{n}{e} \PY{l+s}{\PYZdq{}==\PYZdq{}} \PY{n}{e7} \PY{p}{[}\PY{p}{]}
\end{Verbatim}
\end{tcolorbox}

    \begin{Verbatim}[commandchars=\\\{\}]
Prim ("min", CstI 1, CstI 10)
    \end{Verbatim}

    
    \begin{Verbatim}[commandchars=\\\{\}]
Evaluated using min: 1
    \end{Verbatim}

    
    \begin{Verbatim}[commandchars=\\\{\}]
Prim ("max", CstI 1, CstI 10)
    \end{Verbatim}

    
    \begin{Verbatim}[commandchars=\\\{\}]
Evaluated using max: 10
    \end{Verbatim}

    
    \begin{Verbatim}[commandchars=\\\{\}]
Prim ("==", CstI 66, CstI 66)
    \end{Verbatim}

    
    \begin{Verbatim}[commandchars=\\\{\}]
Evaluated using ==: 1
    \end{Verbatim}

    
    \begin{Verbatim}[commandchars=\\\{\}]
Prim ("==", CstI 66, CstI 22)
    \end{Verbatim}

    
    \begin{Verbatim}[commandchars=\\\{\}]
Evaluated using ==: 0
    \end{Verbatim}

    
    \begin{enumerate}
\def\labelenumi{(\roman{enumi})}
\setcounter{enumi}{2}
\tightlist
\item
  Rewrite one of the eval functions to evaluate the arguments of a
  primitive before branching out on the operator in this style:
\end{enumerate}

\begin{verbatim}
let rec eval e (env : (string * int) list) : int =
    match e with
    | ...
    | Prim(ope, e1, e2) ->
        let i1 = ...
        let i2 = ...
        match ope with
        | "+" -> i1 + i2
        | ...
\end{verbatim}

    \begin{tcolorbox}[breakable, size=fbox, boxrule=1pt, pad at break*=1mm,colback=cellbackground, colframe=cellborder]
\prompt{In}{incolor}{ }{\boxspacing}
\begin{Verbatim}[commandchars=\\\{\}]
\PY{k}{let} \PY{n}{rec} \PY{n}{eval} \PY{n+nf}{e} \PY{p}{(}\PY{n}{env} \PY{p}{:} \PY{p}{(}\PY{k+kt}{string} \PY{p}{*} \PY{k+kt}{int}\PY{p}{)} \PY{n}{list}\PY{p}{)} \PY{p}{:} \PY{k+kt}{int} \PY{p}{=}
    \PY{n}{match} \PY{n}{e} \PY{n}{with}
    \PY{p}{|} \PY{n}{CstI} \PY{n}{i}              \PY{p}{\PYZhy{}}\PY{p}{\PYZgt{}} \PY{n}{i}
    \PY{p}{|} \PY{n}{Var} \PY{n}{x}               \PY{p}{\PYZhy{}}\PY{p}{\PYZgt{}} \PY{n}{lookup} \PY{n}{env} \PY{n}{x} 
    \PY{p}{|} \PY{n}{Prim}\PY{p}{(}\PY{n}{ope}\PY{p}{,} \PY{n}{e1}\PY{p}{,} \PY{n}{e2}\PY{p}{)}   \PY{p}{\PYZhy{}}\PY{p}{\PYZgt{}} 
        \PY{k}{let} \PY{n}{i1} \PY{p}{=} \PY{n}{eval} \PY{n}{e1} \PY{n}{env}
        \PY{k}{let} \PY{n}{i2} \PY{p}{=} \PY{n}{eval} \PY{n}{e2} \PY{n}{env}
        \PY{n}{match} \PY{n}{ope} \PY{n}{with}
        \PY{p}{|} \PY{l+s}{\PYZdq{}+\PYZdq{}}   \PY{p}{\PYZhy{}}\PY{p}{\PYZgt{}} \PY{n}{i1} \PY{p}{+} \PY{n}{i2}
        \PY{p}{|} \PY{l+s}{\PYZdq{}\PYZhy{}\PYZdq{}}   \PY{p}{\PYZhy{}}\PY{p}{\PYZgt{}} \PY{n}{i1} \PY{p}{\PYZhy{}} \PY{n}{i2}
        \PY{p}{|} \PY{l+s}{\PYZdq{}*\PYZdq{}}   \PY{p}{\PYZhy{}}\PY{p}{\PYZgt{}} \PY{n}{i1} \PY{p}{*} \PY{n}{i2}
        \PY{p}{|} \PY{l+s}{\PYZdq{}max\PYZdq{}} \PY{p}{\PYZhy{}}\PY{p}{\PYZgt{}} \PY{k}{if} \PY{n}{i1} \PY{p}{\PYZgt{}} \PY{n}{i2} \PY{n}{then} \PY{n}{i1} \PY{k}{else} \PY{n}{i2}
        \PY{p}{|} \PY{l+s}{\PYZdq{}min\PYZdq{}} \PY{p}{\PYZhy{}}\PY{p}{\PYZgt{}} \PY{k}{if} \PY{n}{i1} \PY{p}{\PYZlt{}} \PY{n}{i2} \PY{n}{then} \PY{n}{i1} \PY{k}{else} \PY{n}{i2}
        \PY{p}{|} \PY{l+s}{\PYZdq{}==\PYZdq{}}  \PY{p}{\PYZhy{}}\PY{p}{\PYZgt{}} \PY{k}{if} \PY{n}{i1} \PY{p}{=} \PY{n}{i2} \PY{n}{then} \PY{l+m}{1} \PY{k}{else} \PY{l+m}{0}
        \PY{p}{|} \PY{n}{\PYZus{}}     \PY{p}{\PYZhy{}}\PY{p}{\PYZgt{}} \PY{n}{failwith} \PY{l+s}{\PYZdq{}unknown primitive\PYZdq{}}
\end{Verbatim}
\end{tcolorbox}

    \begin{enumerate}
\def\labelenumi{(\roman{enumi})}
\setcounter{enumi}{3}
\tightlist
\item
  Extend the expression language with conditional expressions If(e1, e2,
  e3) corresponding to java's e1 ? e2 : e3 or F\#'s if e1 then e2 else
  e3
\end{enumerate}

    \begin{tcolorbox}[breakable, size=fbox, boxrule=1pt, pad at break*=1mm,colback=cellbackground, colframe=cellborder]
\prompt{In}{incolor}{ }{\boxspacing}
\begin{Verbatim}[commandchars=\\\{\}]
\PY{n}{type} \PY{n}{expr} \PY{p}{=} 
  \PY{p}{|} \PY{n}{CstI} \PY{n}{of} \PY{k+kt}{int}
  \PY{p}{|} \PY{n}{Var} \PY{n}{of} \PY{k+kt}{string}
  \PY{p}{|} \PY{n}{Prim} \PY{n}{of} \PY{k+kt}{string} \PY{p}{*} \PY{n}{expr} \PY{p}{*} \PY{n}{expr}
  \PY{p}{|} \PY{n}{If} \PY{n}{of} \PY{n}{expr} \PY{p}{*} \PY{n}{expr} \PY{p}{*} \PY{n}{expr}
\end{Verbatim}
\end{tcolorbox}

    \begin{enumerate}
\def\labelenumi{(\Alph{enumi})}
\setcounter{enumi}{21}
\tightlist
\item
  Extend the interpreter function eval correspondingly. It should
  evaluate e1, and if e1 is non-zero, then evaluate e2 else, evaluate
  e3. You should be able to evaluate the expression If(Var ``a'', CstI
  11, CstI 22) in an enviroment that binds varible a.
\end{enumerate}

    \begin{tcolorbox}[breakable, size=fbox, boxrule=1pt, pad at break*=1mm,colback=cellbackground, colframe=cellborder]
\prompt{In}{incolor}{ }{\boxspacing}
\begin{Verbatim}[commandchars=\\\{\}]
\PY{k}{let} \PY{n}{rec} \PY{n}{eval} \PY{n+nf}{e} \PY{p}{(}\PY{n}{env} \PY{p}{:} \PY{p}{(}\PY{k+kt}{string} \PY{p}{*} \PY{k+kt}{int}\PY{p}{)} \PY{n}{list}\PY{p}{)} \PY{p}{:} \PY{k+kt}{int} \PY{p}{=}
    \PY{n}{match} \PY{n}{e} \PY{n}{with}
    \PY{p}{|} \PY{n}{CstI} \PY{n}{i}              \PY{p}{\PYZhy{}}\PY{p}{\PYZgt{}} \PY{n}{i}
    \PY{p}{|} \PY{n}{Var} \PY{n}{x}               \PY{p}{\PYZhy{}}\PY{p}{\PYZgt{}} \PY{n}{lookup} \PY{n}{env} \PY{n}{x} 
    \PY{p}{|} \PY{n}{Prim}\PY{p}{(}\PY{n}{ope}\PY{p}{,} \PY{n}{e1}\PY{p}{,} \PY{n}{e2}\PY{p}{)}   \PY{p}{\PYZhy{}}\PY{p}{\PYZgt{}} 
        \PY{k}{let} \PY{n}{i1} \PY{p}{=} \PY{n}{eval} \PY{n}{e1} \PY{n}{env}
        \PY{k}{let} \PY{n}{i2} \PY{p}{=} \PY{n}{eval} \PY{n}{e2} \PY{n}{env}
        \PY{n}{match} \PY{n}{ope} \PY{n}{with}
        \PY{p}{|} \PY{l+s}{\PYZdq{}+\PYZdq{}}   \PY{p}{\PYZhy{}}\PY{p}{\PYZgt{}} \PY{n}{i1} \PY{p}{+} \PY{n}{i2}
        \PY{p}{|} \PY{l+s}{\PYZdq{}\PYZhy{}\PYZdq{}}   \PY{p}{\PYZhy{}}\PY{p}{\PYZgt{}} \PY{n}{i1} \PY{p}{\PYZhy{}} \PY{n}{i2}
        \PY{p}{|} \PY{l+s}{\PYZdq{}*\PYZdq{}}   \PY{p}{\PYZhy{}}\PY{p}{\PYZgt{}} \PY{n}{i1} \PY{p}{*} \PY{n}{i2}
        \PY{p}{|} \PY{l+s}{\PYZdq{}max\PYZdq{}} \PY{p}{\PYZhy{}}\PY{p}{\PYZgt{}} \PY{k}{if} \PY{n}{i1} \PY{p}{\PYZgt{}} \PY{n}{i2} \PY{n}{then} \PY{n}{i1} \PY{k}{else} \PY{n}{i2}
        \PY{p}{|} \PY{l+s}{\PYZdq{}min\PYZdq{}} \PY{p}{\PYZhy{}}\PY{p}{\PYZgt{}} \PY{k}{if} \PY{n}{i1} \PY{p}{\PYZlt{}} \PY{n}{i2} \PY{n}{then} \PY{n}{i1} \PY{k}{else} \PY{n}{i2}
        \PY{p}{|} \PY{l+s}{\PYZdq{}==\PYZdq{}}  \PY{p}{\PYZhy{}}\PY{p}{\PYZgt{}} \PY{k}{if} \PY{n}{i1} \PY{p}{=} \PY{n}{i2} \PY{n}{then} \PY{l+m}{1} \PY{k}{else} \PY{l+m}{0}
        \PY{p}{|} \PY{n}{\PYZus{}}     \PY{p}{\PYZhy{}}\PY{p}{\PYZgt{}} \PY{n}{failwith} \PY{l+s}{\PYZdq{}unknown primitive\PYZdq{}}
    \PY{p}{|} \PY{n}{If}\PY{p}{(}\PY{n}{e1}\PY{p}{,} \PY{n}{e2}\PY{p}{,} \PY{n}{e3}\PY{p}{)}    \PY{p}{\PYZhy{}}\PY{p}{\PYZgt{}} 
        \PY{k}{let} \PY{n}{con} \PY{p}{=} \PY{n}{eval} \PY{n}{e1} \PY{n}{env}
        \PY{k}{if} \PY{n}{con} \PY{p}{\PYZgt{}} \PY{l+m}{0} \PY{n}{then} \PY{n}{eval} \PY{n}{e2} \PY{n}{env} \PY{k}{else} \PY{n}{eval} \PY{n}{e3} \PY{n}{env}

\PY{k}{let} \PY{n}{e8} \PY{p}{=} \PY{n}{If}\PY{p}{(}\PY{n}{Var} \PY{l+s}{\PYZdq{}a\PYZdq{}} \PY{p}{,} \PY{n}{CstI} \PY{l+m}{1}\PY{l+m}{1}\PY{p}{,} \PY{n}{CstI} \PY{l+m}{2}\PY{l+m}{2}\PY{p}{)}

\PY{n}{p} \PY{n}{e8}

\PY{n+nf}{display}\PY{p}{(} \PY{n}{sprintf} \PY{l+s}{\PYZdq{}a is \PYZpc{}i\PYZdq{}} \PY{p}{(}\PY{n}{lookup} \PY{n}{env} \PY{l+s}{\PYZdq{}a\PYZdq{}}\PY{p}{)}\PY{p}{)}

\PY{n}{display}\PY{p}{(} \PY{n}{sprintf} \PY{l+s}{\PYZdq{}The result is \PYZpc{}A\PYZdq{}} \PY{p}{(}\PY{n}{eval} \PY{n}{e8} \PY{n}{env}\PY{p}{)}\PY{p}{)}
\end{Verbatim}
\end{tcolorbox}

    \begin{Verbatim}[commandchars=\\\{\}]
If (Var "a", CstI 11, CstI 22)
    \end{Verbatim}

    
    \begin{Verbatim}[commandchars=\\\{\}]
a is 3
    \end{Verbatim}

    
    
    \begin{Verbatim}[commandchars=\\\{\}]
The result is 11
    \end{Verbatim}

    
    Exercise: 1.2 (i) Declare analternative datatype ┬┤aexpr┬┤ for a
representaion of arithmetic expressions without let-bindings. The
datatype should have constructors: CstI, Var, Add, Mul, Sub, for
constants, variables, addition, multiplication, and subtraction. Then
┬┤x * (y + 3) is represented as Mul(Var ``x'', Add(Var ``y'', CstI 3)),
not as Prim(``*``, Var''x'', Prim(``+'', Var ``y'', CstI 3)).

    \begin{tcolorbox}[breakable, size=fbox, boxrule=1pt, pad at break*=1mm,colback=cellbackground, colframe=cellborder]
\prompt{In}{incolor}{ }{\boxspacing}
\begin{Verbatim}[commandchars=\\\{\}]
\PY{n}{type} \PY{n}{aexpr} \PY{p}{=}
\PY{p}{|} \PY{n}{CstI} \PY{n}{of} \PY{k+kt}{int}
\PY{p}{|} \PY{n}{Var} \PY{n}{of} \PY{k+kt}{string}
\PY{p}{|} \PY{n}{Add} \PY{n}{of} \PY{n}{aexpr} \PY{p}{*} \PY{n}{aexpr}
\PY{p}{|} \PY{n}{Mul} \PY{n}{of} \PY{n}{aexpr} \PY{p}{*} \PY{n}{aexpr}
\PY{p}{|} \PY{n}{Sub} \PY{n}{of} \PY{n}{aexpr} \PY{p}{*} \PY{n}{aexpr}
\end{Verbatim}
\end{tcolorbox}

    \begin{enumerate}
\def\labelenumi{(\roman{enumi})}
\setcounter{enumi}{1}
\tightlist
\item
  Write the representation of the expressions. ``v - (w + z)'' and ``2 *
  (v - (w +z))'' and ``x + y + z + v''.
\end{enumerate}

\begin{verbatim}
Sub(Var "v", Add(Var "w", Var "z")),  Mul(CstI 2, Sub(Var "v", Add(Var "w", Var "z"))) and Add(Add(Add(Var "x", Var "y"), Var "z"),Var "v").
\end{verbatim}

    \begin{enumerate}
\def\labelenumi{(\roman{enumi})}
\setcounter{enumi}{2}
\tightlist
\item
  Write an F\# function fmt : aexpr -\textgreater{} string to format
  expressions as strings. For instance, it may format Sub(Var ``x'',
  CstI 34) as the string (x-34). it has very much the same structure as
  the eval function but takes no enviroment argument (because the name
  of a varible is independent of its value).
\end{enumerate}

    \begin{tcolorbox}[breakable, size=fbox, boxrule=1pt, pad at break*=1mm,colback=cellbackground, colframe=cellborder]
\prompt{In}{incolor}{ }{\boxspacing}
\begin{Verbatim}[commandchars=\\\{\}]
\PY{k}{let} \PY{n}{rec} \PY{n}{fmt} \PY{n}{ax} \PY{p}{:} \PY{k+kt}{string} \PY{p}{=} 
    \PY{n}{match} \PY{n}{ax} \PY{n}{with}
    \PY{p}{|} \PY{n}{CstI} \PY{n}{i} \PY{p}{\PYZhy{}}\PY{p}{\PYZgt{}} \PY{k+kt}{string} \PY{n}{i}
    \PY{p}{|} \PY{n}{Var} \PY{n}{x} \PY{p}{\PYZhy{}}\PY{p}{\PYZgt{}} \PY{n}{x}
    \PY{p}{|} \PY{n}{Mul}\PY{p}{(}\PY{n}{x1}\PY{p}{,}\PY{n}{x2}\PY{p}{)} \PY{p}{\PYZhy{}}\PY{p}{\PYZgt{}} \PY{l+s}{\PYZdq{}(\PYZdq{}} \PY{p}{+} \PY{k+kt}{string} \PY{p}{(}\PY{n}{fmt} \PY{n}{x1}\PY{p}{)} \PY{p}{+} \PY{l+s}{\PYZdq{}*\PYZdq{}} \PY{p}{+} \PY{k+kt}{string} \PY{p}{(}\PY{n}{fmt} \PY{n}{x2}\PY{p}{)} \PY{p}{+} \PY{l+s}{\PYZdq{})\PYZdq{}}
    \PY{p}{|} \PY{n}{Add}\PY{p}{(}\PY{n}{x1}\PY{p}{,}\PY{n}{x2}\PY{p}{)} \PY{p}{\PYZhy{}}\PY{p}{\PYZgt{}} \PY{l+s}{\PYZdq{}(\PYZdq{}} \PY{p}{+} \PY{k+kt}{string} \PY{p}{(}\PY{n}{fmt} \PY{n}{x1}\PY{p}{)} \PY{p}{+} \PY{l+s}{\PYZdq{}+\PYZdq{}} \PY{p}{+} \PY{k+kt}{string} \PY{p}{(}\PY{n}{fmt} \PY{n}{x2}\PY{p}{)} \PY{p}{+} \PY{l+s}{\PYZdq{})\PYZdq{}}
    \PY{p}{|} \PY{n}{Sub}\PY{p}{(}\PY{n}{x1}\PY{p}{,}\PY{n}{x2}\PY{p}{)} \PY{p}{\PYZhy{}}\PY{p}{\PYZgt{}} \PY{l+s}{\PYZdq{}(\PYZdq{}} \PY{p}{+} \PY{k+kt}{string} \PY{p}{(}\PY{n}{fmt} \PY{n}{x1}\PY{p}{)} \PY{p}{+} \PY{l+s}{\PYZdq{}\PYZhy{}\PYZdq{}} \PY{p}{+} \PY{k+kt}{string} \PY{p}{(}\PY{n}{fmt} \PY{n}{x2}\PY{p}{)} \PY{p}{+} \PY{l+s}{\PYZdq{})\PYZdq{}}
    \PY{c+c1}{//| \PYZus{} \PYZhy{}\PYZgt{} \PYZdq{}unknown type\PYZdq{}}

\PY{k}{let} \PY{n}{t} \PY{p}{=} \PY{n}{Sub}\PY{p}{(}\PY{n}{Var} \PY{l+s}{\PYZdq{}v\PYZdq{}}\PY{p}{,} \PY{n}{Add}\PY{p}{(}\PY{n}{Var} \PY{l+s}{\PYZdq{}w\PYZdq{}}\PY{p}{,} \PY{n}{Var} \PY{l+s}{\PYZdq{}z\PYZdq{}}\PY{p}{)}\PY{p}{)}
\PY{k}{let} \PY{n}{te} \PY{p}{=} \PY{n}{fmt} \PY{n}{t}
\PY{n}{printf} \PY{l+s}{\PYZdq{}\PYZpc{}s\PYZdq{}} \PY{n}{te}

\PY{k}{let} \PY{n}{t1} \PY{p}{=} \PY{n}{Mul}\PY{p}{(}\PY{n}{CstI} \PY{l+m}{2}\PY{p}{,} \PY{n}{Sub}\PY{p}{(}\PY{n}{Var} \PY{l+s}{\PYZdq{}v\PYZdq{}}\PY{p}{,} \PY{n}{Add}\PY{p}{(}\PY{n}{Var} \PY{l+s}{\PYZdq{}w\PYZdq{}}\PY{p}{,} \PY{n}{Var} \PY{l+s}{\PYZdq{}z\PYZdq{}}\PY{p}{)}\PY{p}{)}\PY{p}{)}
\PY{k}{let} \PY{n}{t1e} \PY{p}{=} \PY{n}{fmt} \PY{n}{t1}
\PY{n}{printf} \PY{l+s}{\PYZdq{}\PYZpc{}s\PYZdq{}} \PY{n}{t1e}

\PY{k}{let} \PY{n}{t2} \PY{p}{=} \PY{n}{Add}\PY{p}{(}\PY{n}{Add}\PY{p}{(}\PY{n}{Add}\PY{p}{(}\PY{n}{Var} \PY{l+s}{\PYZdq{}x\PYZdq{}}\PY{p}{,} \PY{n}{Var} \PY{l+s}{\PYZdq{}y\PYZdq{}}\PY{p}{)}\PY{p}{,} \PY{n}{Var} \PY{l+s}{\PYZdq{}z\PYZdq{}}\PY{p}{)}\PY{p}{,}\PY{n}{Var} \PY{l+s}{\PYZdq{}v\PYZdq{}}\PY{p}{)}
\PY{k}{let} \PY{n}{t2e} \PY{p}{=} \PY{n}{fmt} \PY{n}{t2}
\PY{n}{printf} \PY{l+s}{\PYZdq{}\PYZpc{}s\PYZdq{}} \PY{n}{t2e}

\PY{k}{let} \PY{n}{t4} \PY{p}{=} \PY{n}{Sub}\PY{p}{(}\PY{n}{Var} \PY{l+s}{\PYZdq{}x\PYZdq{}}\PY{p}{,} \PY{n}{CstI} \PY{l+m}{3}\PY{l+m}{4}\PY{p}{)} 
\PY{k}{let} \PY{n}{t4e} \PY{p}{=} \PY{n}{fmt} \PY{n}{t4}
\PY{n}{printf} \PY{l+s}{\PYZdq{}\PYZpc{}s\PYZdq{}} \PY{n}{t4e}
\end{Verbatim}
\end{tcolorbox}

    \begin{Verbatim}[commandchars=\\\{\}]
(v-(w+z))(2*(v-(w+z)))(((x+y)+z)+v)(x-34)
    \end{Verbatim}

    \begin{enumerate}
\def\labelenumi{(\roman{enumi})}
\setcounter{enumi}{3}
\tightlist
\item
  Write an F\# function simplify : aexpr -\textgreater{} aexpr to
  perform expression simplification. For instance, it should simplify
  (x+0) to x and (1+0) to 1. The more ambitius student might want to
  simplify (1+0)*(x+0) to x. Hint: Pattern matching is your friend.
  Don't forget the case where you cannot simplify anything.
\end{enumerate}

see. p.~10 in PLC for more.

    \begin{tcolorbox}[breakable, size=fbox, boxrule=1pt, pad at break*=1mm,colback=cellbackground, colframe=cellborder]
\prompt{In}{incolor}{ }{\boxspacing}
\begin{Verbatim}[commandchars=\\\{\}]
\PY{k}{let} \PY{n}{rec} \PY{n}{simplify} \PY{n}{ax} \PY{p}{:} \PY{n}{aexpr} \PY{p}{=}
    \PY{n}{match} \PY{n}{ax} \PY{n}{with}
    \PY{p}{|}\PY{n}{Add}\PY{p}{(}\PY{n}{x1}\PY{p}{,}\PY{n}{x2}\PY{p}{)} \PY{p}{\PYZhy{}}\PY{p}{\PYZgt{}} 
        \PY{k}{if} \PY{n}{x1}\PY{p}{=} \PY{n}{CstI} \PY{l+m}{0} \PY{p}{|}\PY{p}{|} \PY{n}{x2}\PY{p}{=} \PY{n}{CstI} \PY{l+m}{0} \PY{n}{then} 
            \PY{k}{if} \PY{n}{x1}\PY{p}{=} \PY{n}{CstI} \PY{l+m}{0} \PY{n}{then} \PY{n}{simplify} \PY{n}{x2} \PY{k}{else} \PY{n}{simplify} \PY{n}{x1} 
        \PY{k}{else} \PY{n+nf}{Add}\PY{p}{(}\PY{n}{simplify} \PY{n}{x1}\PY{p}{,} \PY{n}{simplify} \PY{n}{x2}\PY{p}{)}
    \PY{p}{|}\PY{n}{Sub}\PY{p}{(}\PY{n}{x1}\PY{p}{,}\PY{n}{x2}\PY{p}{)} \PY{p}{\PYZhy{}}\PY{p}{\PYZgt{}} 
        \PY{k}{if} \PY{n}{x1} \PY{p}{=} \PY{n}{x2} \PY{p}{|}\PY{p}{|} \PY{n}{x2} \PY{p}{=} \PY{n}{CstI} \PY{l+m}{0} \PY{n}{then} 
            \PY{k}{if} \PY{n}{x2} \PY{p}{=} \PY{n}{CstI} \PY{l+m}{0} \PY{n}{then} \PY{n}{simplify} \PY{n}{x1} \PY{k}{else} \PY{n}{CstI} \PY{l+m}{0}
        \PY{k}{else} \PY{n+nf}{Sub}\PY{p}{(}\PY{n}{simplify} \PY{n}{x1}\PY{p}{,} \PY{n}{simplify} \PY{n}{x2}\PY{p}{)}
    \PY{p}{|}\PY{n}{Mul}\PY{p}{(}\PY{n}{x1}\PY{p}{,}\PY{n}{x2}\PY{p}{)} \PY{p}{\PYZhy{}}\PY{p}{\PYZgt{}}
        \PY{k}{if} \PY{n}{x1} \PY{p}{=} \PY{n}{CstI} \PY{l+m}{1} \PY{p}{|}\PY{p}{|} \PY{n}{x2} \PY{p}{=} \PY{n}{CstI} \PY{l+m}{1} \PY{n}{then} 
            \PY{k}{if} \PY{n}{x1} \PY{p}{=} \PY{n}{CstI} \PY{l+m}{1} \PY{n}{then} \PY{n}{simplify} \PY{n}{x2} \PY{k}{else} \PY{n}{simplify} \PY{n}{x1}
        \PY{k}{else} 
            \PY{k}{if} \PY{n}{x1} \PY{p}{=} \PY{n}{CstI} \PY{l+m}{0} \PY{p}{|}\PY{p}{|} \PY{n}{x2} \PY{p}{=} \PY{n}{CstI} \PY{l+m}{0} \PY{n}{then} \PY{n}{CstI} \PY{l+m}{0} \PY{k}{else} \PY{n}{Mul}\PY{p}{(}\PY{n}{simplify} \PY{n}{x1}\PY{p}{,} \PY{n}{simplify} \PY{n}{x2}\PY{p}{)}
    \PY{p}{|}\PY{n}{\PYZus{}} \PY{p}{\PYZhy{}}\PY{p}{\PYZgt{}} \PY{n}{ax}

\PY{c+c1}{// e + 0 \PYZhy{}\PYZgt{} e}
\PY{k}{let} \PY{n}{t5} \PY{p}{=} \PY{n}{Add}\PY{p}{(}\PY{n}{CstI} \PY{l+m}{0}\PY{p}{,} \PY{n}{CstI} \PY{l+m}{1}\PY{p}{)}
\PY{k}{let} \PY{n}{t5s} \PY{p}{=} \PY{n}{simplify} \PY{n}{t5}
\PY{n}{printf} \PY{l+s}{\PYZdq{}\PYZpc{}A\PYZdq{}} \PY{n}{t5s}

\PY{c+c1}{// 0 + e \PYZhy{}\PYZgt{} e}
\PY{k}{let} \PY{n}{t6} \PY{p}{=} \PY{n}{Add}\PY{p}{(}\PY{n}{CstI} \PY{l+m}{1}\PY{p}{,} \PY{n}{CstI} \PY{l+m}{0}\PY{p}{)}
\PY{k}{let} \PY{n}{t6s} \PY{p}{=} \PY{n}{simplify} \PY{n}{t6}
\PY{n}{printf} \PY{l+s}{\PYZdq{}\PYZpc{}A\PYZdq{}} \PY{n}{t6s}

\PY{c+c1}{// Should not change}
\PY{k}{let} \PY{n}{t1s} \PY{p}{=} \PY{n}{simplify} \PY{n}{t1}
\PY{n}{printf} \PY{l+s}{\PYZdq{}\PYZpc{}A\PYZdq{}} \PY{n}{t1s}

\PY{c+c1}{// e \PYZhy{} 0 \PYZhy{}\PYZgt{} e}
\PY{k}{let} \PY{n}{t7} \PY{p}{=} \PY{n}{Sub}\PY{p}{(}\PY{n}{CstI} \PY{l+m}{1}\PY{p}{,} \PY{n}{CstI} \PY{l+m}{0}\PY{p}{)}
\PY{k}{let} \PY{n}{t7s} \PY{p}{=} \PY{n}{simplify} \PY{n}{t7}
\PY{n}{printf} \PY{l+s}{\PYZdq{}\PYZpc{}A\PYZdq{}} \PY{n}{t7s}

\PY{c+c1}{// e \PYZhy{} e \PYZhy{}\PYZgt{} 0}
\PY{k}{let} \PY{n}{t8} \PY{p}{=} \PY{n}{Sub}\PY{p}{(}\PY{n}{CstI} \PY{l+m}{5}\PY{p}{,} \PY{n}{CstI} \PY{l+m}{5}\PY{p}{)}
\PY{k}{let} \PY{n}{t8s} \PY{p}{=} \PY{n}{simplify} \PY{n}{t8}
\PY{n}{printf} \PY{l+s}{\PYZdq{}\PYZpc{}A\PYZdq{}} \PY{n}{t8s}

\PY{c+c1}{// 1 * e \PYZhy{}\PYZgt{} e}
\PY{k}{let} \PY{n}{t9} \PY{p}{=} \PY{n}{Mul}\PY{p}{(}\PY{n}{CstI} \PY{l+m}{1}\PY{p}{,} \PY{n}{CstI} \PY{l+m}{4}\PY{l+m}{2}\PY{p}{)}
\PY{k}{let} \PY{n}{t9s} \PY{p}{=} \PY{n}{simplify} \PY{n}{t9}
\PY{n}{printf} \PY{l+s}{\PYZdq{}\PYZpc{}A\PYZdq{}} \PY{n}{t9s}

\PY{c+c1}{// e * 1 \PYZhy{}\PYZgt{} e}
\PY{k}{let} \PY{n}{t10} \PY{p}{=} \PY{n}{Mul}\PY{p}{(}\PY{n}{CstI} \PY{l+m}{4}\PY{l+m}{2}\PY{p}{,} \PY{n}{CstI} \PY{l+m}{1}\PY{p}{)}
\PY{k}{let} \PY{n}{t10s} \PY{p}{=} \PY{n}{simplify} \PY{n}{t10}
\PY{n}{printf} \PY{l+s}{\PYZdq{}\PYZpc{}A\PYZdq{}} \PY{n}{t10s}

\PY{c+c1}{// Since negatives is not handled in the simplify function it will reduce to (0\PYZhy{}1) + 1 AND not 0}
\PY{k}{let} \PY{n}{t11} \PY{p}{=} \PY{n}{Mul}\PY{p}{(}\PY{n}{Add}\PY{p}{(}\PY{n}{Sub}\PY{p}{(}\PY{n}{CstI} \PY{l+m}{0}\PY{p}{,} \PY{n}{CstI} \PY{l+m}{1}\PY{p}{)}\PY{p}{,} \PY{n}{Add}\PY{p}{(}\PY{n}{CstI} \PY{l+m}{0}\PY{p}{,} \PY{n}{CstI} \PY{l+m}{1}\PY{p}{)}\PY{p}{)}\PY{p}{,} \PY{n}{CstI} \PY{l+m}{1}\PY{p}{)}
\PY{k}{let} \PY{n}{t11f} \PY{p}{=} \PY{n}{fmt} \PY{n}{t11}
\PY{n}{printf} \PY{l+s}{\PYZdq{}\PYZpc{}A\PYZdq{}} \PY{n}{t11f}
\PY{k}{let} \PY{n}{t11s} \PY{p}{=} \PY{n}{simplify} \PY{n}{t11}
\PY{n}{printf} \PY{l+s}{\PYZdq{}\PYZpc{}A\PYZdq{}} \PY{n}{t11s}
\PY{k}{let} \PY{n}{t11sf} \PY{p}{=} \PY{n}{fmt} \PY{n}{t11s}
\PY{n}{printf} \PY{l+s}{\PYZdq{}\PYZpc{}A\PYZdq{}} \PY{n}{t11sf}
\end{Verbatim}
\end{tcolorbox}

    \begin{Verbatim}[commandchars=\\\{\}]
CstI 1CstI 1Mul (CstI 2, Sub (Var "v", Add (Var "w", Var "z")))CstI 1CstI 0CstI
42CstI 42"(((0-1)+(0+1))*1)"Add (Sub (CstI 0, CstI 1), CstI 1)"((0-1)+1)"
    \end{Verbatim}


    % Add a bibliography block to the postdoc
    
    
    
\end{document}
